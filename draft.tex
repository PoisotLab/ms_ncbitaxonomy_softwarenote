%!TEX TS-program = xelatex
\documentclass[11pt]{article}

\usepackage[english]{babel}

\usepackage{amsmath,amssymb,amsfonts}
\usepackage[utf8]{inputenc}
\usepackage[T1]{fontenc}
\usepackage{stix2}
\usepackage[scaled]{helvet}
\usepackage[scaled]{inconsolata}

\usepackage{lastpage}

\usepackage{setspace}

\usepackage{ccicons}

\usepackage[hang,flushmargin]{footmisc}

\usepackage{geometry}

\setlength{\parindent}{0pt}
\setlength{\parskip}{6pt plus 2pt minus 1pt}

\usepackage{fancyhdr}
\renewcommand{\headrulewidth}{0pt}\providecommand{\tightlist}{%
  \setlength{\itemsep}{0pt}\setlength{\parskip}{0pt}}

\makeatletter
\newcounter{tableno}
\newenvironment{tablenos:no-prefix-table-caption}{
  \caption@ifcompatibility{}{
    \let\oldthetable\thetable
    \let\oldtheHtable\theHtable
    \renewcommand{\thetable}{tableno:\thetableno}
    \renewcommand{\theHtable}{tableno:\thetableno}
    \stepcounter{tableno}
    \captionsetup{labelformat=empty}
  }
}{
  \caption@ifcompatibility{}{
    \captionsetup{labelformat=default}
    \let\thetable\oldthetable
    \let\theHtable\oldtheHtable
    \addtocounter{table}{-1}
  }
}
\makeatother

\usepackage{array}
\newcommand{\PreserveBackslash}[1]{\let\temp=\\#1\let\\=\temp}
\let\PBS=\PreserveBackslash

\usepackage[breaklinks=true]{hyperref}
\hypersetup{colorlinks,%
citecolor=blue,%
filecolor=blue,%
linkcolor=blue,%
urlcolor=blue}
\usepackage{url}

\usepackage{caption}
\setcounter{secnumdepth}{0}
\usepackage{cleveref}

\usepackage{graphicx}
\makeatletter
\def\maxwidth{\ifdim\Gin@nat@width>\linewidth\linewidth
\else\Gin@nat@width\fi}
\makeatother
\let\Oldincludegraphics\includegraphics
\renewcommand{\includegraphics}[1]{\Oldincludegraphics[width=\maxwidth]{#1}}

\usepackage{longtable}
\usepackage{booktabs}

\usepackage{color}
\usepackage{fancyvrb}
\newcommand{\VerbBar}{|}
\newcommand{\VERB}{\Verb[commandchars=\\\{\}]}
\DefineVerbatimEnvironment{Highlighting}{Verbatim}{commandchars=\\\{\}}
% Add ',fontsize=\small' for more characters per line
\usepackage{framed}
\definecolor{shadecolor}{RGB}{248,248,248}
\newenvironment{Shaded}{\begin{snugshade}}{\end{snugshade}}
\newcommand{\KeywordTok}[1]{\textcolor[rgb]{0.13,0.29,0.53}{\textbf{#1}}}
\newcommand{\DataTypeTok}[1]{\textcolor[rgb]{0.13,0.29,0.53}{#1}}
\newcommand{\DecValTok}[1]{\textcolor[rgb]{0.00,0.00,0.81}{#1}}
\newcommand{\BaseNTok}[1]{\textcolor[rgb]{0.00,0.00,0.81}{#1}}
\newcommand{\FloatTok}[1]{\textcolor[rgb]{0.00,0.00,0.81}{#1}}
\newcommand{\ConstantTok}[1]{\textcolor[rgb]{0.00,0.00,0.00}{#1}}
\newcommand{\CharTok}[1]{\textcolor[rgb]{0.31,0.60,0.02}{#1}}
\newcommand{\SpecialCharTok}[1]{\textcolor[rgb]{0.00,0.00,0.00}{#1}}
\newcommand{\StringTok}[1]{\textcolor[rgb]{0.31,0.60,0.02}{#1}}
\newcommand{\VerbatimStringTok}[1]{\textcolor[rgb]{0.31,0.60,0.02}{#1}}
\newcommand{\SpecialStringTok}[1]{\textcolor[rgb]{0.31,0.60,0.02}{#1}}
\newcommand{\ImportTok}[1]{#1}
\newcommand{\CommentTok}[1]{\textcolor[rgb]{0.56,0.35,0.01}{\textit{#1}}}
\newcommand{\DocumentationTok}[1]{\textcolor[rgb]{0.56,0.35,0.01}{\textbf{\textit{#1}}}}
\newcommand{\AnnotationTok}[1]{\textcolor[rgb]{0.56,0.35,0.01}{\textbf{\textit{#1}}}}
\newcommand{\CommentVarTok}[1]{\textcolor[rgb]{0.56,0.35,0.01}{\textbf{\textit{#1}}}}
\newcommand{\OtherTok}[1]{\textcolor[rgb]{0.56,0.35,0.01}{#1}}
\newcommand{\FunctionTok}[1]{\textcolor[rgb]{0.00,0.00,0.00}{#1}}
\newcommand{\VariableTok}[1]{\textcolor[rgb]{0.00,0.00,0.00}{#1}}
\newcommand{\ControlFlowTok}[1]{\textcolor[rgb]{0.13,0.29,0.53}{\textbf{#1}}}
\newcommand{\OperatorTok}[1]{\textcolor[rgb]{0.81,0.36,0.00}{\textbf{#1}}}
\newcommand{\BuiltInTok}[1]{#1}
\newcommand{\ExtensionTok}[1]{#1}
\newcommand{\PreprocessorTok}[1]{\textcolor[rgb]{0.56,0.35,0.01}{\textit{#1}}}
\newcommand{\AttributeTok}[1]{\textcolor[rgb]{0.77,0.63,0.00}{#1}}
\newcommand{\RegionMarkerTok}[1]{#1}
\newcommand{\InformationTok}[1]{\textcolor[rgb]{0.56,0.35,0.01}{\textbf{\textit{#1}}}}
\newcommand{\WarningTok}[1]{\textcolor[rgb]{0.56,0.35,0.01}{\textbf{\textit{#1}}}}
\newcommand{\AlertTok}[1]{\textcolor[rgb]{0.94,0.16,0.16}{#1}}
\newcommand{\ErrorTok}[1]{\textcolor[rgb]{0.64,0.00,0.00}{\textbf{#1}}}
\newcommand{\NormalTok}[1]{#1}

\newlength{\cslhangindent}
\setlength{\cslhangindent}{1.5em}
\newlength{\csllabelwidth}
\setlength{\csllabelwidth}{3em}
\newenvironment{CSLReferences}[3] % #1 hanging-ident, #2 entry spacing
 {% don't indent paragraphs
  \setlength{\parindent}{0pt}
  % turn on hanging indent if param 1 is 1
  \ifodd #1 \everypar{\setlength{\hangindent}{\cslhangindent}}\ignorespaces\fi
  % set entry spacing
  \ifnum #2 > 0
  \setlength{\parskip}{#2\baselineskip}
  \fi
 }%
 {}
\usepackage{calc} % for \widthof, \maxof
\newcommand{\CSLBlock}[1]{#1\hfill\break}
\newcommand{\CSLLeftMargin}[1]{\parbox[t]{\maxof{\widthof{#1}}{\csllabelwidth}}{#1}}
\newcommand{\CSLRightInline}[1]{\parbox[t]{\linewidth}{#1}}
\newcommand{\CSLIndent}[1]{\hspace{\cslhangindent}#1}\geometry{verbose,letterpaper,tmargin=2.2cm,bmargin=2.2cm,lmargin=2.2cm,rmargin=2.2cm}

\usepackage{lineno}
\usepackage[nolists,noheads]{endfloat}

\pagestyle{plain}

\tolerance=1
\emergencystretch=\maxdimen
\hyphenpenalty=10000
\hbadness=10000

\doublespacing

\fancypagestyle{normal}
{
  \fancyhf{}
  \fancyfoot[R]{\footnotesize\sffamily\thepage\ of \pageref*{LastPage}}
}
\begin{document}
\raggedright
\thispagestyle{empty}
{\Large\bfseries\sffamily NCBITaxonomy.jl - rapid biological names
finding and reconciliation}
\vskip 5em

%
\href{https://orcid.org/0000-0002-0735-5184}{Timothée\,Poisot}%
%
\,\textsuperscript{1,2}\quad %
\href{https://orcid.org/0000-0002-0965-1649}{Rory\,Gibb}%
%
\,\textsuperscript{3,4}\quad %
\href{https://orcid.org/0000-0002-4308-6321}{Sadie J.\,Ryan}%
%
\,\textsuperscript{5,6,7}\quad %
\href{https://orcid.org/0000-0001-6960-8434}{Colin J.\,Carlson}%
%
\,\textsuperscript{8,9}

\textsuperscript{1}\,Université de Montréal, Départment de Sciences
Biologiques, Montréal QC, Canada\quad \textsuperscript{2}\,Québec Centre
for Biodiversity Science, Montréal, QC,
Canada\quad \textsuperscript{3}\,Centre on Climate Change and Planetary
Health, London School of Hygiene and Tropical Medicine, London,
UK\quad \textsuperscript{4}\,Centre for Mathematical Modelling of
Infectious Diseases, London School of Hygiene and Tropical Medicine,
London, UK\quad \textsuperscript{5}\,Emerging Pathogens Institute,
University of Florida, Gainesville, FL, United States of
America\quad \textsuperscript{6}\,School of Life Sciences, University of
KwaZulu-Natal, Durban, South Africa\quad \textsuperscript{7}\,Department
of Geography, University of Florida, Gainesville, FL, United States of
America\quad \textsuperscript{8}\,Department of Microbiology and
Immunology, Georgetown University Medical Center, Georgetown University,
Washington, D.C., United States of
America\quad \textsuperscript{9}\,Center for Global Health Science and
Security, Georgetown University Medical Center, Georgetown University,
Washington, D.C., United States of America


\textbf{Correspondance to:}\\
Timothée Poisot --- \texttt{timothee.poisot@umontreal.ca}\\

\vfill
This work is released by its authors under a CC-BY 4.0 license\hfill\ccby\\
Last revision: \emph{\today}

\clearpage
\thispagestyle{empty}

\vfill
\texttt{NCBITaxonomy.jl} is a package designed to facilitate the
reconciliation and cleaning of taxonomic names, using a local copy of
the NCBI taxonomic backbone (Federhen 2012, Schoch et al. 2020); The
basic search functions are coupled with quality-of-life functions
including case-insensitive search and custom fuzzy string matching to
facilitate the amount of information that can be extracted automatically
while allowing efficient manual curation and inspection of results.
\texttt{NCBITaxonomy.jl} works with version 1.6 of the Julia programming
language (Bezanson et al. 2017), and relies on the Apache Arrow format
to store a local copy of the NCBI raw taxonomy files. The design of
\texttt{NCBITaxonomy.jl} has been inspired by similar efforts, like the
R package \texttt{taxadb} (Norman et al. 2020), which provides an
offline alternative to packages like \texttt{taxize} (Chamberlain and
Szöcs 2013).



\vfill

\clearpage
\linenumbers
\pagestyle{normal}

Unambiguously identifying species is a far more challenging task than it
may appear. There are a vast number of reasons for this. Different
databases keep different taxonomic ``backbones,'' \emph{i.e.} different
data structures in which names are mapped to species, and organised in a
hierarchy. Not all names are unique identifiers to groups. For example,
\emph{Io} can either refer to a genus of plants from the aster family,
or to a genus of molluscs; the genus \emph{Mus} (of which the house
mouse \emph{Mus musculus} is a species), contains a sub-genus
\emph{also} named \emph{Mus}. Conversely, the same species can have
several names, which are valid synonyms: for example, the domestic cow
\emph{Bos taurus} admits \emph{Bos primigenius taurus} as a valid
synonym. Taxonomic nomenclature also changes regularly, with groups
being split, merged, or moved to a new position in the tree of life;
this is, notably, a common occurrence with viral taxonomy, each
subsequent version of which can differ markedly from the last; compare,
\emph{e.g} Lefkowitz et al. (2018) to Walker et al. (2020).

To add to the complexity, one must also consider that most taxa names
are at some point manually typed, which has the potential to introduce
additional mistakes in raw data; it is likely to expect that such
mistakes may arise when attempting to write down the (perfectly valid)
names of the bacterial isolate known as \emph{Myxococcus
llanfairpwllgwyngyllgogerychwyrndrobwllllantysiliogogogochensis}, or of
the crowned slaty flycatcher \emph{Griseotyrannus
aurantioatrocristatus}. These mistakes are more likely when dealing with
hyper-diverse samples, like plant census (Dauncey et al. 2016, Wagner
2016, Conti et al. 2021). In addition to binomial names, the same
species can be known by many vernacular (common) names, which are
language or even region-specific: \emph{Ovis aries}, for example, has
valid English vernaculars including lamb, sheep, wild sheep, and
domestic sheep.

All these considerations are actually important when matching species
names both within and across datasets. Let us consider the following
species survey of individual fishes, European chub, \emph{Cyprinus
cephalus}, \emph{Leuciscus cephalus}, \emph{Squalius cephalus}: all are
the same species (\emph{S. cephalus}), referred to as one of the
vernacular (European chub) and two formerly accepted names now
classified as synonyms. A cautious estimate of diversity based on the
user-supplied names would give \(n=4\) species, when there is in fact
only one.

A package with the ability to handle the sources of errors outlined
above, and especially while provide an authoritative classification, can
accelerate the work of consuming large volumes of biodiversity data. For
example, this package was used in the process of developing the
\emph{CLOVER} database (Gibb et al. 2021) of host-virus associations, by
reconciling the names of viruses and mammals from four different
sources, where all of the issues described above were present.

\hypertarget{overview-of-functionalities}{%
\section{Overview of
functionalities}\label{overview-of-functionalities}}

An up-to-date version of the documentation for \texttt{NCBITaxonomy.jl}
can be found online at
\url{https://docs.ecojulia.org/NCBITaxonomy.jl/stable/}, including
examples and a documentation of every method. The package is released
under the MIT license. Contributions can be made in the form of issues
(bug reports, questions, features suggestions) and pull requests. The
package can be checked out and installed anonymously from the central
Julia repository:

\begin{Shaded}
\begin{Highlighting}[]
\KeywordTok{using}\NormalTok{ Pkg}

\CommentTok{\# This line should go in the Julia configuration file {-} note that the path}
\CommentTok{\# will be created if it doesn\textquotesingle{}t exist, and will be used to store the}
\CommentTok{\# raw taxonomic table}
\ConstantTok{ENV}\NormalTok{[}\StringTok{"NCBITAXONOMY\_PATH"}\NormalTok{] }\OperatorTok{=}\NormalTok{ joinpath(homedir()}\OperatorTok{,} \StringTok{"data"}\OperatorTok{,} \StringTok{"NCBITaxonomy.jl"}\NormalTok{)}

\NormalTok{Pkg.add(}\StringTok{"NCBITaxonomy"}\NormalTok{) }\CommentTok{\# Dowloading the files may take a long time}
\end{Highlighting}
\end{Shaded}

The package will download the most recent version of the NCBI taxonomy
database, and transform in into a set of Apache Arrow files ready for
use. Note that the \texttt{NCBITAXONOMY\_PATH} can specified on a
per-project basis, and as long as the package is not re-built, the local
set of tables downloaded from NCBI will not change; this way, users can
re-run an analysis with a guarantee that the underlying taxonomic
backbone has not changed.

\hypertarget{improved-name-matching}{%
\subsection{Improved name matching}\label{improved-name-matching}}

Name finding is primarily done through the \texttt{taxon} function,
which admits either a unique NCBI identifier (\emph{e.g.}
\texttt{taxon(36219)} for the bogue \emph{Boops boops}), a string
(\texttt{taxon("Boops\ boops")}), or a data frame with a restricted list
of names (see the next section). The \texttt{taxon} method has
additional arguments to perform fuzzy matching in order to catch
possible typos (\texttt{taxon("Boops\ bops";\ strict=false)}), to
perform a lowercase search (useful when alphanumeric codes are part of
the taxon name, like for some viruses), and to restrict the the search
to a specific taxonomic rank.

The lowercase search can be a preferable alternative to fuzzy string
matching. Consider the string \texttt{Adeno-associated\ virus\ 3b} - it
has three names with equal distance (under the Levensthein string
distance function):

\begin{Shaded}
\begin{Highlighting}[]
\NormalTok{julia}\OperatorTok{\textgreater{}}\NormalTok{ similarnames(}\StringTok{"Adeno{-}associated virus 3b"}\OperatorTok{;}\NormalTok{ threshold}\OperatorTok{=}\FloatTok{0.95}\NormalTok{)}
\FloatTok{3}\OperatorTok{{-}}\NormalTok{element }\DataTypeTok{Vector}\NormalTok{\{}\DataTypeTok{Pair}\NormalTok{\{NCBITaxon}\OperatorTok{,} \DataTypeTok{Float64}\NormalTok{\}\}}\OperatorTok{:}
\NormalTok{  Adeno}\OperatorTok{{-}}\NormalTok{associated virus }\OperatorTok{{-}} \FloatTok{3}\NormalTok{ (ncbi}\OperatorTok{:}\FloatTok{46350}\NormalTok{) }\OperatorTok{=\textgreater{}} \FloatTok{0.96}
\NormalTok{   Adeno}\OperatorTok{{-}}\NormalTok{associated virus }\FloatTok{3}\NormalTok{B (ncbi}\OperatorTok{:}\FloatTok{68742}\NormalTok{) }\OperatorTok{=\textgreater{}} \FloatTok{0.96}
\NormalTok{ Adeno}\OperatorTok{{-}}\NormalTok{associated virus }\FloatTok{3}\NormalTok{A (ncbi}\OperatorTok{:}\FloatTok{1406223}\NormalTok{) }\OperatorTok{=\textgreater{}} \FloatTok{0.96}
\end{Highlighting}
\end{Shaded}

Depending on the operating system, either of these three names can be
returned; compare to the output of a case insensitive name search:

\begin{Shaded}
\begin{Highlighting}[]
\NormalTok{julia}\OperatorTok{\textgreater{}}\NormalTok{ taxon(}\StringTok{"Adeno{-}associated virus 3b"}\OperatorTok{;}\NormalTok{ casesensitive}\OperatorTok{=}\ExtensionTok{false}\NormalTok{)}
\NormalTok{Adeno}\OperatorTok{{-}}\NormalTok{associated virus }\FloatTok{3}\NormalTok{B (ncbi}\OperatorTok{:}\FloatTok{68742}\NormalTok{)}
\end{Highlighting}
\end{Shaded}

This returns the correct name.

\hypertarget{name-matching-output-and-error-handling}{%
\subsection{Name matching output and error
handling}\label{name-matching-output-and-error-handling}}

The \texttt{taxon} function will either return a \texttt{NCBITaxon}
object (made of a \texttt{name} and \texttt{id}), or throw either a
\texttt{NameHasNoDirectMatch} (with instructions about how to possible
solve it, using the \texttt{similarnames} function), or a
\texttt{NameHasMultipleMatches} (listing the possible valid matches, and
suggesting to use \texttt{alternativetaxa} to find the correct one).
Therefore, the common way to work with the \texttt{taxon} function would
be to wrap it in a \texttt{try}/\texttt{catch} statement:

\begin{Shaded}
\begin{Highlighting}[]
\KeywordTok{try}
\NormalTok{  taxon(name)}
  \CommentTok{\# Additional operations with the matched name}
\KeywordTok{catch}\NormalTok{ err}
  \KeywordTok{if}\NormalTok{ isa(err}\OperatorTok{,}\NormalTok{ NameHasNoDirectMatch)}
    \CommentTok{\# What to do if no match is found}
  \KeywordTok{elseif}\NormalTok{ isa(err}\OperatorTok{,}\NormalTok{ NameHasMultipleMatches)}
    \CommentTok{\# What to do if there are multiple matches}
  \KeywordTok{else}
    \CommentTok{\# What to do in case of another error that is not NCBITaxonomy specific}
  \KeywordTok{end}
\KeywordTok{end}
\end{Highlighting}
\end{Shaded}

These functions will not demand any user input in the form of key
presses (though they can be wrapped in additional code to allow it), as
they are intended to run on clusters without supervision. The
\texttt{taxon} function has good scaling using muliple threads. For
convenience in rapidly getting a taxon for demonstration purposes, we
also provide a string macro, whereby \emph{e.g.}
\texttt{ncbi"Procyon\ lotor"} will return the taxon object for the
raccoon.

\hypertarget{name-filtering-functions}{%
\subsection{Name filtering functions}\label{name-filtering-functions}}

As the full NCBI names table has over 3 million entries at the time of
writing, we have provided a number of functions to restrict the scope of
names that are searched. These are driven by the NCBI \emph{divisions}.
For example \texttt{nf\ =\ mammalfilter(true)} will return a data frame
containing the names of mammals, inclusive of rodents and primates, and
can be used with \emph{e.g.} \texttt{taxon(nf,\ "Pan")}. This has the
dual advantage of making search faster, but also of avoiding matching on
names that are shared by another taxonomic group (which is not an issue
with \emph{Pan}, but is an issue with \emph{e.g.} \emph{Io} as mentioned
in the introduction).

Note that the use of a restricted list of names can have significant
performance consequences: compare, for example, the time taken to return
the taxon \emph{Pan} (ID 9596) in the entire database, in all mammals,
and in all primates:

\begin{longtable}[]{@{}lclll@{}}
\toprule
Names list & Fuzzy matching & Time (ms) & Allocations & Memory
allocated\tabularnewline
\midrule
\endhead
all & no & 23 & 34 & 2 KiB\tabularnewline
& yes & 105 & 2580 & 25 MiB\tabularnewline
\texttt{mammalfilter(true)} & no & 0.55 & 32 & 2 KiB\tabularnewline
& yes & 1.9 & 551 & 286 KiB\tabularnewline
\texttt{primatefilter()} & no & 0.15 & 33 & 2 KiB\tabularnewline
& yes & 0.3 & 92 & 27 KiB\tabularnewline
\bottomrule
\end{longtable}

Clearly, the optimal search strategy is to (i) rely on name filters to
ensure that search are conducted within the appropriate NCBI division,
and (ii) only rely on fuzzy matching when the strict or lowercase match
fails to return a name, as fuzzy matching can result in order of
magnitude more run time and memory footprint. These numbers were
obtained on a single Intel i7-8665U CPU (@ (1.90GHz). Using
\texttt{"chimpanzees"} as the search string (the NCBI recognized
vernacular for \emph{Pan}) gave qualitatively similar results,
suggesting that there is no performance cost associated with working
with synonyms or verncular input data.

\hypertarget{quality-of-life-functions}{%
\subsection{Quality of life functions}\label{quality-of-life-functions}}

In order to facilitate working with names, we provide the
\texttt{authority} function (gives the full taxonomic authority for a
name), \texttt{synonyms} (to get alternative valid names),
\texttt{vernacular} (for English common names), and \texttt{rank} (for
the taxonomic rank).

\hypertarget{taxonomic-lineages-navigation}{%
\subsection{Taxonomic lineages
navigation}\label{taxonomic-lineages-navigation}}

The \texttt{children} function will return all nodes that are directly
descended from a taxon; the \texttt{descendants} function will
recursively apply this function to all descendants of these nodes, until
only terminal leaves are reached. The \texttt{parent} function is an
``upwards'' equivalent, giving that taxon from which a taxon descents;
the \texttt{lineage} function chains calls to \texttt{parent} until
either \texttt{taxon(1)} (the taxonomy root) or an arbitrary ancestor is
reached.

The \texttt{taxonomicdistance} function (and its in-place equivalent,
\texttt{taxonomicdistance!}, which uses memory-efficient re-allocation
if the user needs to change the distance between taxonomic ranks) uses
the Shimatani (2001) approach to reconstruct a matrix of distances based
on taxonomy, which can serve as a rough proxy when no phylogenies are
available.

\textbf{Acknowledgements:} This work was supported by funding to the
Viral Emergence Research Initiative (VERENA) consortium including NSF
BII 2021909 and a grant from Institut de Valorisation des Données
(IVADO), by the NSERC Discovery Grants and Discovery Acceleration
Supplement programs, and by a donation from the Courtois Foundation.
Benchmarking of this package on distributed systems was enabled by
support provided by Calcul Québec (\texttt{www.calculquebec.ca}) and
Compute Canada (\texttt{www.computecanada.ca}). TP wrote the initial
code, TP and CJC contributed to API design, and all authors contributed
to functionalities and usability testing.

\hypertarget{references}{%
\section*{References}\label{references}}
\addcontentsline{toc}{section}{References}

\hypertarget{refs}{}
\begin{CSLReferences}{1}{0}
\leavevmode\hypertarget{ref-Bezanson2017JulFre}{}%
Bezanson, J. et al. 2017. Julia: A Fresh Approach to Numerical
Computing. - SIAM Review 59: 65--98.

\leavevmode\hypertarget{ref-Chamberlain2013TaxTax}{}%
Chamberlain, S. A. and Szöcs, E. 2013. Taxize: Taxonomic search and
retrieval in R. - F1000Research 2: 191.

\leavevmode\hypertarget{ref-Conti2021MatAlg}{}%
Conti, M. et al. 2021. Match Algorithms for Scientific Names in
FlorItaly, the Portal to the Flora of Italy. - Plants 10: 974.

\leavevmode\hypertarget{ref-Dauncey2016ComMis}{}%
Dauncey, E. A. et al. 2016. Common mistakes when using plant names and
how to avoid them. - European Journal of Integrative Medicine 8:
597--601.

\leavevmode\hypertarget{ref-Federhen2012NcbTax}{}%
Federhen, S. 2012. The NCBI taxonomy database. - Nucleic acids research
40: D136--D143.

\leavevmode\hypertarget{ref-Gibb2021DatPro}{}%
Gibb, R. et al. 2021. Data proliferation, reconciliation, and synthesis
in viral ecology. - bioRxiv: 2021.01.14.426572.

\leavevmode\hypertarget{ref-Lefkowitz2018VirTax}{}%
Lefkowitz, E. J. et al. 2018. Virus taxonomy: The database of the
International Committee on Taxonomy of Viruses (ICTV). - Nucleic Acids
Research 46: D708--D717.

\leavevmode\hypertarget{ref-Norman2020TaxHig}{}%
Norman, K. E. A. et al. 2020. Taxadb: A high-performance local taxonomic
database interface. - Methods in Ecology and Evolution 11: 1153--1159.

\leavevmode\hypertarget{ref-Schoch2020NcbTax}{}%
Schoch, C. L. et al. 2020. NCBI Taxonomy: A comprehensive update on
curation, resources and tools. - Database in press.

\leavevmode\hypertarget{ref-Shimatani2001MeaSpe}{}%
Shimatani, K. 2001. On the Measurement of Species Diversity
Incorporating Species Differences. - Oikos 93: 135--147.

\leavevmode\hypertarget{ref-Wagner2016RevSof}{}%
Wagner, V. 2016. A review of software tools for spell-checking taxon
names in vegetation databases. - Journal of Vegetation Science 27:
1323--1327.

\leavevmode\hypertarget{ref-Walker2020ChaVir}{}%
Walker, P. J. et al. 2020. Changes to virus taxonomy and the Statutes
ratified by the International Committee on Taxonomy of Viruses (2020). -
Archives of Virology 165: 2737--2748.

\end{CSLReferences}

\end{document}
